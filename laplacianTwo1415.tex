\documentclass{article}[12pt,a4paper]
\usepackage{amsmath}
\usepackage{amssymb}
\usepackage{graphicx}
\usepackage{amsthm}
\usepackage{multirow}
\usepackage{mathtools}
\usepackage{mathrsfs}
\usepackage{preview}
\usepackage{caption}
\usepackage{xcolor}
% \usepackage[dvipsnames]{xcolor}
% \usepackage[a4paper]{geometry} 
% % % % % % % % % % % % % % % % % % % % % % % % % % % % % % % % % % %
%a4paper: height = 11.69in, width=8.27in
%  height=844.84pt, width=597.67pt
\setlength{\textheight}{609pt} 
\setlength{\textwidth}{424pt}
\setlength{\oddsidemargin}{18pt} 
% \setlength{\evensidemargin}{0pt}
\setlength{\hoffset}{0pt} 
\setlength{\voffset}{0pt} 
\setlength{\topmargin}{0pt} 
\setlength{\headheight}{12pt} 
\setlength{\headsep}{12pt} 
\setlength{\marginparsep}{11pt}
\setlength{\marginparwidth}{54pt} 
\setlength{\marginparpush}{5pt}
\setlength{\footskip}{30pt}
% % % % % % % % % % % % % % % % % % % % % % % % % % % % % % % % % % %
\renewcommand{\baselinestretch}{1.2}
%\renewcommand{\baselinestretch}{1.5}
% % % % % % % % % % % % % % % % % % % % % % % % % % % % % % % % % % %
% % % % % % % % % % % % % % % % % % % % % % % % % % % % % % % % % % %
\begin{document}
% codefile2020.tex
\setlength{\parindent}{0pt}
% % % % % % % % % % % % % % % % % % % % % % % % % % % % % % % % % % % % % % % % % %
% symbols
\newcommand{\mR}{\mathbb{R}}
\newcommand{\mC}{\mathbb{C}}
\newcommand{\mS}{\mathbb{S}}
\newcommand{\mZ}{\mathbb{Z}}
\newcommand{\mH}{\mathbb{H}}
\newcommand{\lie}{\mathcal{L}}
% vectors
\newcommand{\bi}{{\bf i}}
\newcommand{\bj}{{\bf j}}
\newcommand{\bk}{{\bf k}}
\newcommand{\bn}{{\bf n}}
\newcommand{\br}{{\bf r}}
\newcommand{\bu}{{\bf u}}
\newcommand{\bv}{{\bf v}}
\newcommand{\bw}{{\bf w}}
\newcommand{\bx}{{\bf x}}
\newcommand{\by}{{\bf y}}
\newcommand{\bz}{{\bf z}}
\newcommand{\vi}{\vec{i}}
\newcommand{\vj}{\vec{j}}
\newcommand{\vk}{\vec{k}}
% complex conjugate
\newcommand{\oa}{\overline{a}}
\newcommand{\ob}{\overline{b}}
\newcommand{\oc}{\overline{c}}
\newcommand{\of}{\overline{f}}
\newcommand{\og}{\overline{g}}
\newcommand{\oh}{\overline{h}}
\newcommand{\oi}{\overline{i}}
\newcommand{\oj}{\overline{j}}
\newcommand{\ok}{\overline{k}}
\newcommand{\ol}{\overline{l}}
\newcommand{\om}{\overline{m}}
\newcommand{\on}{\overline{n}}
\newcommand{\op}{\overline{p}}
\newcommand{\oq}{\overline{q}}
\newcommand{\os}{\overline{s}}
\newcommand{\ot}{\overline{t}}
\newcommand{\ou}{\overline{u}}
\newcommand{\ov}{\overline{v}}
\newcommand{\ow}{\overline{w}}
\newcommand{\ox}{\overline{x}}
\newcommand{\oy}{\overline{y}}
\newcommand{\oz}{\overline{z}}
% overline greeks
\newcommand{\oalpha}{\overline{\alpha}}
\newcommand{\obeta}{\overline{\beta}}
\newcommand{\ogamma}{\overline{\gamma}}
\newcommand{\olambda}{\overline{\lambda}}
\newcommand{\omu}{\overline{\mu}}
\newcommand{\oomega}{\overline{\omega}}
\newcommand{\opsi}{\overline{\psi}}
\newcommand{\otheta}{\overline{\theta}}
\newcommand{\ozeta}{\overline{\zeta}}
% math symbol
\newcommand{\bl}{\backslash}
\newcommand{\prt}{\partial}
\newcommand{\oprt}{\overline{\partial}}
\newcommand{\curl}{\mbox{curl}}
\newcommand{\mdiv}{\mbox{div}}
\newcommand{\hess}{\mbox{Hess}}
\newcommand{\nab}{\nabla}
\newcommand{\wdg}{\wedge}
% % % % % % % % % % % % % % % % % % % % % % % % % % % % % % % % % % % % % % % % % %
% \newcommand and \def
\newcommand{\dvec}[1]{\frac{d}{d#1}}
\newcommand{\ddvec}[2]{\frac{d#1}{d#2}}
\newcommand{\tvec}[1]{\frac{\prt}{\prt #1}}
\newcommand{\ttvec}[2]{\frac{\prt #1}{\prt #2}}
\newcommand{\ttves}[2]{\frac{\prt^2 #1}{\prt #2}}
\newcommand{\pvec}[1]{\prt_{#1}}
\newcommand{\define}{\overset{{\scriptscriptstyle \Delta}}{=}}
\newcommand{\mysum}[3][n]{\sum_{#1=#2}^{#3}}
\def\ba#1{\begin{array}{#1}}
\def\ea{\end{array}}
\def\bt#1{\begin{tabular}{#1}}
\def\et{\end{tabular}}
% spacing
\newcommand{\spa}{\mbox{ }}
\newcommand{\man}{\quad \mbox{and} \quad}
\newcommand{\mor}{\quad \mbox{or} \quad}
\newcommand{\ds}{\displaystyle}
\newcommand{\spts}{\scriptstyle}
\newcommand{\mcl}{\mbox{ $|$ }}
\newcommand{\bvline}{\biggr\rvert}
\newcommand{\hl}{\hspace{-0.1in}}
% % % % % % % % % % % % % % % % % % % % % % % % % % % % % % % % % % % % % % % % % %
% \newthereom
\newtheorem*{prop}{Proposition} 
\newtheorem*{thm}{Theorem}  
\newtheorem*{den}{Definition}
\newtheorem*{lem}{Lemma} 
\newtheorem*{coll}{Corollary}
% equation environment
\newcounter{eqtncount}
\newenvironment{eqtn}
{\stepcounter{eqtncount}
	\addtocounter{equation}{-1}
	\renewcommand\theequation{\arabic{eqtncount}}\equation}
{\endequation}
% % % % % % % % % % % % % % % % % % % % % % % % % % % % % % % % % % % % % % % % % %
\pagestyle{empty}
\begin{center} Laplacian Operator (II) \end{center}

\begin{center} {\sc 4. Third covariant derivatives of function}\end{center}
Let $(M, g)$ be an $n$-dimensional Riemannian manifold. 
Let $\{e_1, e_2, \cdots, e_n\}$ be an orthonormal frame on $M$. 
Suppose $f$ is a smooth real-valued function on $M$. 
Recall that the first and second covariant derivatives of $f$ are defined by
$$ \nabla f \,=\, f_j \, e_j \man 
\nabla^2 f \,=\, f_{ij}\, \omega_j \otimes e_i \,=\, \big(df_i + f_j\, \omega_{ji}\big)\otimes e_i $$
Let $S=\nabla^2 f$. Note that 
$$ (\nabla_X S)(Y) \,=\, \nabla_X\big(S(Y)\big) - S(\nabla_X Y) \,=\, 
	\nabla_X \nabla_Y(\nabla f) \,-\, \nabla_{\nabla_X Y}(\nabla f)\,,$$
$$ (\nabla_Y S)(X) \,=\, \nabla_Y\big(S(X)\big) - S(\nabla_Y X) \,=\, 
	\nabla_Y \nabla_X(\nabla f) \,-\, \nabla_{\nabla_Y X}(\nabla f)\,. $$
Therefore,
\begin{equation}
(\nabla_X S)(Y) \,-\, (\nabla_Y S)(X) \,=\, R(X, Y)\,\nabla f\,.
\end{equation}
\vspace*{0.1in}

We are computing for $\nabla S$ as follows. \\[0.1in]
$\begin{array}{rcl}
\nabla S &=& \ds \nabla \big(f_{ij}\, \omega_j \otimes e_i\big) \\[0.1in] 
&=& \ds 
	df_{ij}\, \omega_j \otimes e_i \,+\, f_{ij}\, d\omega_j \otimes e_i 
	\,-\, f_{ij}\, \omega_j \wedge \nabla e_i \\[0.1in]
&=& \ds 
	df_{ij}\, \omega_j \otimes e_i \,+\, f_{ij}\, \big(\omega_{jk} \wedge \omega_k\big) \otimes e_i
	\,-\, f_{ij}\, \omega_j \wedge \big( \omega_{ik} \otimes e_k\big) \\[0.1in]
&=& \ds 
	\big( df_{ij} \,+\, f_{ik}\, \omega_{kj} \,+\, f_{kj}\, \omega_{ki} \big)\,\wedge\, \omega_j \otimes e_i \,.
\end{array}$ \\[0.1in]

If we write $\nabla^3 f \,=\, f_{ijk}\,\omega_k \wedge \omega_j \otimes e^i$, then we have
\begin{equation}
f_{ijk}\,\omega_k \,=\,  df_{ij} \,+\, f_{ip}\, \omega_{pj} \,+\, f_{qj}\, \omega_{qi} \,.
\end{equation}
\vspace*{0.1in}

In the following, we justify the equalities
$$ f_{ij}\,=\, g\big(\nabla_{e_j}(\nabla f)\,,\,e_i\big)
\man f_{ijk} = g\big((\nabla S)(e_k, e_j)\,,\,e_i\big)\,.$$
They clarify that the definition of covariant derivatives here match our usual understanding. \\[0.1in]
$\begin{array}{rcl}
\nabla_{e_j}(\nabla f) &=& \ds \nabla_{e_j}\big(f_k\, e_k\big) \\[0.1in]
&=& \ds 
	e_j\,e_k(f)\, e_k \,+\, f_k \, \nabla_{e_j}e_k \\[0.1in]
&=& \ds 
	e_j\,e_k(f)\, e_k \,+\, f_k \, \Gamma_{jk}^l\, e_l \\[0.1in]
&=& \ds
	\big(e_j\, e_i(f) \,+\, f_k\, \Gamma_{jk}^i\big)\, e_i \\[0.1in]
&=& \ds 
	f_{ij}\, e_i 
\end{array}$\\[0.2in]
So the first identity is proven. Moreover, we have \\
\newpage

$\begin{array}{rcl}
(\nabla S)(e_k, e_j) &=& \ds \big(\nabla_{e_k}(\nabla^2 f)\big)(e_j) \\[0.1in]
&=& \ds 
	\nabla_{e_k}\big(\nabla_{e_j}(\nabla f)\big) \,-\, (\nabla^2 f)(\nabla_{e_k}e_j) \\[0.1in]
&=& \ds 
	\nabla_{e_k}\big(f_{ij}\, e_i\big) \,-\, \Gamma_{kj}^l\, \nabla_{e_l}(\nabla f) \\[0.1in]
&=& \ds 
	e_k(f_{ij})\, e_i \,+\, f_{ij}\, \Gamma_{ki}^l\, e_l \,-\, \Gamma_{kj}^l \, f_{pl}\, e_p \\[0.1in]
&=& \ds 
	\big( e_k(f_{ij}) \,+\, f_{pj}\, \Gamma_{kp}^i \,-\, f_{ip}\, \Gamma_{kj}^p \big) \, e_i \\[0.1in]
&=& \ds 
	\big( df_{ij}(e_k) \,+\, f_{pj}\,\omega_{pi}(e_k) \,+\, f_{ip}\, \omega_{pj}(e_k)\big) \, e_i \\[0.1in] 
&=& \ds 
	f_{ijk}\, e_i \,.
\end{array}$\\[0.2in]
Therefore, the second identity also holds. In equation (1), we could put $X=e_k$ and $Y=e_j$. 
$$ (\nabla_{e_k} S)(e_j) \,-\, (\nabla_{e_j} S)(e_k) \,=\, R(e_k, e_j)\,\nabla f $$
$$ \implies\quad f_{ijk}\, e_i \,-\, f_{ikj}\, e_i \,=\, f_p \, R_{kjpq}\, e_q $$
By comparison, we have 
\begin{equation} 
f_{ijk} \,-\, f_{ikj} \,=\,  f_p \, R_{ipjk} \,.
\end{equation}
This identity is called the Ricci identity.
\vspace*{0.2in}
% % % % % % % % % % % % % % % % % % % % % % % % % % % % % % % % % % %

\begin{center} {\sc 5. Covariant derivatives of differential forms}\end{center}
The covariant derivatives of the covectors $\omega_j$ in the dual frame can be found by:
$$ (\nabla_X \omega_j)(e_i) \,=\, X\big(\omega_j(e_i)\big) \,-\, \omega_j(\nabla_X e_i)
\,=\, -\omega_j(\nabla_X e_i) \,. $$
So we have \,$\nabla_X(\omega_j)\,=\, -\omega_j(\nabla_X e_i)\, \omega_i$\, and hence
$$ \nabla(\omega_j) \,=\, -\Gamma_{ki}^j\, \omega_k \otimes \omega_i \,=\, 
	-\omega_{ij} \otimes \omega_i \,=\, \omega_{ji} \otimes \omega_i\,. $$

Let $\omega$ be a $p$-form on the $n$-dimensional manifold $M$. We may let 
$$ \omega \,=\, \sum_{i_1,\cdots,i_p=1}^n\, a_{i_1\cdots i_{p-1} i_p}\, \omega_{i_p}\wedge\omega_{i_{p-1}}\wedge\cdots\wedge\omega_{i_1}\,,$$
assuming that \,$a_{i_1\cdots i_{p-1} i_p} \neq 0$\, only when \,$i_p<i_{p-1}<\cdots<i_1$\,. \\

For any vector $X$, we have \\[0.1in]
$\begin{array}{rcl}
\nabla_X \omega &=& \ds 
	da_{i_1\cdots i_p}(X)\, \omega_{i_p}\wedge\cdots\wedge\omega_{i_1}
	\,+\, a_{i_1\cdots i_p} \,\big(\nabla_X(\omega_{i_p})\big)\wedge\omega_{i_{p-1}}\wedge\cdots\wedge\omega_{i_1} \\[0.1in]
&& \ds 
	\,+\, a_{i_1\cdots i_p}\, \omega_{i_p}\wedge \big(\nabla_X(\omega_{i_{p-1}})\big)\wedge\cdots\wedge\omega_{i_1} \\[0.1in]
&& \ds
 	\,+\, \cdots \,+\, a_{i_1\cdots i_p}\, \omega_{i_p}\wedge\omega_{i_{p-1}}\wedge \cdots\wedge 
	\big( \nabla_X(\omega_{i_1})\big) \\[0.2in]
&=& \ds
	da_{i_1\cdots i_p}(X)\,\omega_{i_p}\wedge\cdots\wedge\omega_{i_1} 
	\,+\, a_{i_1\cdots i_p}\,\omega_{i_pj_p}(X)\, \omega_{j_p}\wedge\omega_{i_{p-1}}\wedge\cdots\wedge\omega_{i_1} \\[0.1in]
&& \ds
	\,+\, a_{i_1\cdots i_p}\,\omega_{i_{p-1}j_{p-1}}(X)\, \omega_{i_p}\wedge\omega_{j_{p-1}}\wedge\cdots\wedge\omega_{i_1} \\[0.1in]
&& \ds 
	\,+\, \cdots \,+\,
	a_{i_1\cdots i_p}\, \omega_{i_1j_1}(X)\, \omega_{i_p}\wedge\omega_{i_{p-1}}\wedge\cdots\wedge \omega_{j_1} \,.
\end{array}$ \\
\newpage

We would write
$$ \nabla_X \omega \,=\, 
	\big(da_{i_1\cdots i_p} + \sum_{r, j_r} a_{i_1\cdots j_r\cdots i_p}\, \omega_{j_r i_r}\big)(X) 
	\,\,\omega_{i_p}\wedge\cdots\wedge\omega_{i_1}\,.  $$

Therefore, 
\begin{equation}
	\nabla\omega \,=\, \Big(da_{i_1\cdots i_p} \,+\, \sum_{r, j_r} a_{i_1\cdots j_r\cdots i_p} \, \omega_{j_r i_r}\Big)
	\otimes \omega_{i_p}\wedge \cdots \wedge \omega_{i_1}\,.
\end{equation}
\vspace*{0.1in}

If we let \,$\nabla \omega \,=\, a_{i_1\cdots i_p,j}\,\omega_j 
	\otimes \big(\omega_{i_p} \wedge \cdots \wedge \omega_{i_1}\big) $\,,
it results in
$$ a_{i_1 \cdots i_p,j}\, \omega_j \,=\, da_{i_1\cdots i_p} \,+\, \sum_{r, j_r} a_{i_1\cdots j_r\cdots i_p} \, \omega_{j_r i_r}\,. $$
The coefficient $a_{i_1\cdots i_p,j}$ is found by
$$ a_{i_1\cdots i_p, j} \,=\, e_j\big(a_{i_1\cdots i_p}\big) \,+\, \sum_{r, j_r} a_{i_1\cdots j_r\cdots i_p} \, \Gamma_{jj_r}^{i_r}\,. $$
\vspace*{0.1in}

For the moment, we may define every coefficient $a_{j_1\cdots j_p}$ by
$$ a_{j_1\cdots j_p} \,=\, \det\Big[\omega_{j_r}(e_{i_s})\Big] \, a_{i_1\cdots i_p} $$
where \,$(i_1,\cdots,i_p)$\, is a rearrangement of \,$(j_1,\cdots,j_p)$\, such that
\,$i_p<i_{p-1}<\cdots<i_1$\,. As a result
$$  \omega \,=\, a_{i_1\cdots i_p}\, \omega_{i_p}\wedge\cdots\wedge\omega_{i_1}\,.$$
Note that we have
$$ \omega\big(e_{j_p}, \cdots, e_{j_1}\big) \,=\, \det\Big[\omega_{j_r}(e_{i_s})\Big]\, \omega\big(e_{i_p},\cdots, e_{i_1}\big)
\,=\, \det\Big[\omega_{j_r}(e_{i_s})\Big]\, a_{i_1\cdots i_p} \,=\, a_{j_1\cdots j_p}\,. $$

Therefore, \\[0.1in]
$\begin{array}{rl}
&\ds (\nabla_{e_j}\omega)\big(e_{i_p}, e_{i_{p-1}}, \cdots, e_{i_1}\big) \\[0.1in]
=&\ds 
	e_j\Big(\omega\big(e_{i_p}, \cdots, e_{i_1}\big) \Big) \,-\,
	\sum_{r=1}^p \omega\Big(e_{i_p},\cdots, \nabla_{e_j}e_{i_r}, \cdots, e_{i_1}\Big) \\[0.2in]
=& \ds 
	e_k\big( a_{j_1\cdots j_p}\big) \,-\, \Gamma_{ji_r}^{j_r}\, a_{i_1\cdots j_r \cdots i_p} \\[0.1in]
=& \ds 
	e_k\big( a_{j_1\cdots j_p}\big) \,+\, \Gamma_{jj_r}^{i_r}\, a_{i_1\cdots j_r \cdots i_p} \\[0.1in]
=& \ds 
	a_{i_1\cdots i_p, j} \,.
\end{array}$ \\[0.1in]
In the symmetric representation of a differential form, we validate that 
$$ a_{i_1\cdots i_p, j} \,=\, (\nabla_{e_j}\omega)\big(e_{i_p}, e_{i_{p-1}}, \cdots, e_{i_1}\big)\,. $$

\newpage
% % % % % % % % % % % % % % % % % % % % % % % % % % % % % % % % % % %
If we define \,$\ds \omega \,=\, \sum a_{i_1\cdots i_p}\, \omega_{i_p} \wedge\cdots\wedge \omega_{i_1}$\, note that\\[0.1in]
$\begin{array}{rcl}
d\omega &=& \ds \sum_{i_1,\cdots,i_p} da_{i_1\cdots i_p} \wedge 
	\omega_{i_p} \wedge\cdots\wedge \omega_{i_1} \\[0.2in]
&& \ds 
	+\, \sum_{i_1,\cdots,i_p} \sum_{r} (-1)^{p-r}\, a_{i_1\cdots i_p}\,
	\omega_{i_p} \wedge\cdots\wedge d\omega_{i_r} \wedge\cdots\wedge \omega_{i_1}\\[0.3in]
&=& \ds 
	\sum da_{i_1\cdots i_p} \wedge \omega_{i_p} \wedge\cdots\wedge \omega_{i_1} \\[0.15in]
&& \ds 
	+\, \sum (-1)^{p-r}\, a_{i_1\cdots i_r \cdots i_p}\, \omega_{i_rj_r} \wedge \omega_{j_r} \wedge
	\omega_{i_p} \wedge\cdots\wedge \widetilde{\omega_{i_r}} \wedge\cdots\wedge \omega_{i_1}\\[0.3in]
&=& \ds 
	\sum da_{i_1\cdots i_p} \wedge \omega_{i_p} \wedge\cdots\wedge \omega_{i_1} \\[0.15in]
&& \ds 
	+\, \sum a_{i_1\cdots i_r \cdots i_p}\, \omega_{i_rj_r} \wedge
	\omega_{i_p} \wedge\cdots\wedge \omega_{j_r} \wedge\cdots\wedge \omega_{i_1} \\[0.3in]
&=& \ds
	\sum \Big(da_{i_1\cdots i_p} \,+\, \sum a_{i_1\cdots j_r \cdots i_p}\, \omega_{j_ri_r}\Big)
	\wedge \, \omega_{i_p} \wedge\cdots\wedge \omega_{i_1} \\[0.3in]
&=& \ds 
	\sum a_{i_1\cdots i_p, k}\, \omega_k \wedge \omega_{i_p} \wedge\cdots\wedge \omega_{i_1}\,. \\
\end{array}$ \\[0.1in]

As a result, we have 
\begin{equation*}
d\omega \,=\, \sum_{i_1,\cdots,i_p,k} a_{i_1\cdots i_p, k}\, \omega_k \wedge \omega_{i_p} \wedge\cdots\wedge \omega_{i_1}\,.
\end{equation*}

\newpage
% % % % % % % % % % % % % % % % % % % % % % % % % % % % % % % % % % %
For any vector field $Y$, $\nabla_Y \omega$\,  is a $p$-form on $M$. 
The second covariant derivative of $\omega$ is found by
$$ \big(\nabla_X \nabla \omega\big)(Y) \,=\, \nabla_X (\nabla_Y \omega) \,-\, \nabla_{\nabla_X Y}\,\omega\,. $$
Compute for the first term. \\[0.1in]
$\begin{array}{rl}
& \Big(\nabla_X (\nabla_Y \omega)\Big)(u_1, \cdots, u_p) \\[0.1in]
=& \ds
	X\Big((\nabla_Y \omega)(u_1, \cdots, u_p)\Big) 
	\,-\, \sum_{r=1}^p (\nabla_Y \omega)\big(u_1, \cdots, \nabla_X u_r, \cdots, u_p\big) \\[0.2in]
=&\ds 
	X\Big(Y\big(\omega(u_1, \cdots, u_p)\big)\Big)
	\,-\, \sum_{s=1}^p X\Big(\omega\big(u_1, \cdots, \nabla_Y u_s, \cdots, u_p \big)\Big) \\[0.1in]
&\ds 
	\,-\, \sum_{r=1}^p Y\Big(\omega\big(u_1, \cdots, \nabla_X u_r, \cdots, u_p\big)\Big) 
	\,+\, \sum_{r\neq s} \omega\big(u_1, \cdots, \nabla_X u_r, \cdots, \nabla_Y u_s, \cdots, u_p\big) \\[0.1in]
&\ds 
	\,+\, \sum_{r=1}^p \omega\big(u_1, \cdots, \nabla_Y \nabla_X u_r, \cdots, u_p\big) 
\end{array}$ \\[0.1in]

Compute for the second term.\\[0.1in]
$\begin{array}{rl}
& \big(\nabla_{\nabla_X Y}\,\omega\big)(u_1, \cdots, u_p) \\[0.1in]
=& \ds 
	(\nabla_X Y)\Big(\omega(u_1, \cdots, u_p)\Big) \,-\, \sum_{r=1}^p \omega\big(u_1, \cdots, \nabla_{\nabla_X Y}u_r, \cdots, u_p\big)
\end{array}$ \\[0.1in]

Note that by symmetry, we have \\[0.1in]
$\begin{array}{rl}
&\ds \Big(\nabla_X (\nabla_Y \omega)\,-\, \nabla_Y(\nabla_X \omega)\Big)(u_1, \cdots, u_p) \\[0.1in]
=& \ds 
	[X, Y]\Big(\omega(u_1, \cdots, u_p)\Big)
	\,+\, \sum_{r=1}^p \omega\Big(u_1, \cdots, \big(\nabla_Y \nabla_X - \nabla_X \nabla_Y\big)u_r, \cdots, u_p\Big) \,.
\end{array}$ \\[0.1in]

Moreover, we obtain\\[0.1in]
$ \begin{array}{rl}
&\ds \Big(\nabla_{\nabla_X Y}\,\omega - \nabla_{\nabla_Y X}\,\omega \Big)(u_1, \cdots, u_p) \\[0.1in]
=& \ds 
	[X, Y]\Big(\omega(u_1, \cdots, u_p)\Big) 
	\,-\,  \sum_{r=1}^p \omega\big(u_1, \cdots, \nabla_{[X,Y]}u_r, \cdots, u_p\big)\,.
\end{array}$ \\[0.1in]

Therefore, \\[0.1in]
$ \begin{array}{rl}
&\ds \Big((\nabla_X \nabla \omega)(Y) \,-\, (\nabla_Y \nabla \omega)(X)\Big)(u_1, \cdots, u_p) \\[0.2in]
=& \ds \sum_{r=1}^p \omega\Big(u_1, \cdots, \big(\nabla_Y \nabla_X - \nabla_X \nabla_Y + \nabla_{[X, Y]}\big)\,u_r, \cdots, u_p\Big) \\[0.2in]
=& \ds \sum_{r=1}^p \omega\Big(u_1, \cdots, R(Y,X)u_r, \cdots, u_p\Big)\,. \\[0.2in]
\end{array}$ \\
\newpage

We summary this result as Equation (6).
\begin{equation}
\Big((\nabla_X \nabla \omega)(Y) \,-\, (\nabla_Y \nabla \omega)(X)\Big)(u_1, \cdots, u_p) \,=\,
\sum_{r=1}^p \omega\Big(u_1, \cdots, R(Y,X)u_r, \cdots, u_p\Big)\,.
\end{equation}

We are now ready to compute for the coefficients of the tensor $\nabla^2 \omega$. Let $\omega$ be the $p$-form
$$ \omega \,=\, a_{i_1\cdots i_p}\, \omega_{i_p}\wedge \cdots\wedge\omega_{i_1}\,. $$
$\nabla \omega$\, can be expressed as 
$$ \nabla \omega \,=\, a_{i_1\cdots i_p,j}\,\omega_j \otimes 
	\big(\omega_{i_p}\wedge \omega_{i_{p-1}} \wedge \cdots \wedge \omega_{i_1}\big)\,. $$

Take the covariant derivative of $\nabla \omega$. \\[0.2in]
$\begin{array}{rcl}
\nabla\big(\nabla \omega\big) &=& \ds 
	da_{i_1\cdots i_p,j} \wedge \omega_j \otimes 
	\big(\omega_{i_p} \wedge \cdots \wedge \omega_{i_1}\big) 
	\,+\, a_{i_1\cdots i_p,j}\, d\omega_j \otimes \big(\omega_{i_p} \wedge \cdots \wedge \omega_{i_1}\big) \\[0.1in]
&& \ds 
	\,-\, a_{i_1\cdots i_p,j}\,\omega_j \,\wedge\, \nabla\big(\omega_{i_p} \wedge \cdots \wedge \omega_{i_1} \big) \\[0.2in]
&=& 
	\ds 
	da_{i_1\cdots i_p,j} \wedge \omega_j \otimes \big(\omega_{i_p} \wedge \cdots \wedge \omega_{i_1}\big) 
	\,+\, a_{i_1\cdots i_p,j}\, \omega_{jk}\wedge \omega_k \otimes \big(\omega_{i_p} \wedge \cdots \wedge \omega_{i_1}\big) \\[0.1in]
&& \ds
	\,-\, \sum_{r, j_r} a_{i_1\cdots i_p,j}\,\omega_j \wedge \omega_{i_rj_r}
	\otimes \Big(\omega_{i_p} \wedge \cdots \wedge \omega_{j_r} \wedge \cdots \wedge \omega_{i_1}\Big) \\[0.2in]
&=& \ds 
	da_{i_1\cdots i_p,j} \wedge \omega_j \otimes \big(\omega_{i_p} \wedge \cdots \wedge \omega_{i_1}\big) 
	\,+\, a_{i_1\cdots i_p,s}\, \omega_{sj} \wedge \omega_j \otimes \big(\omega_{i_p} \wedge \cdots \wedge \omega_{i_1}\big) \\[0.1in]
&& \ds 
	\,+\, \sum_{r, j_r} a_{i_1 \cdots j_r \cdots i_p,j}\, \omega_{j_r i_r}\wedge \omega_j 
	\otimes  \big(\omega_{i_p} \wedge \cdots \wedge \omega_{i_1}\big) \\[0.2in]
&=& \ds 
	\Big(da_{i_1\cdots i_p,j} \,+\, a_{i_1\cdots i_p,s}\, \omega_{sj} 
	\,+\, \sum_{r,j_r} a_{i_1 \cdots j_r \cdots i_p,j}\, \omega_{j_r i_r} \Big)
	\,\wedge\, \omega_j \otimes \big(\omega_{i_p} \wedge \cdots \wedge \omega_{i_1}\big) \,.
\end{array}$ \\[0.1in]

If we write \,$\nabla^2 \omega \,=\, a_{i_1\cdots i_p,jk}\,\,\omega_k \wedge \omega_j 
\otimes \big(\omega_{i_p} \wedge \cdots \wedge \omega_{i_1}\big)$\,, then we have
\begin{equation}
a_{i_1\cdots i_p,jk}\,\omega_k \,=\, da_{i_1\cdots i_p,j} \,+\, a_{i_1\cdots i_p,s}\, \omega_{sj} 
	\,+\, \sum_{r,j_r} a_{i_1 \cdots j_r \cdots i_p,j}\, \omega_{j_r i_r} \,.
\end{equation}

If we assume that \,$a_{i_1\cdots i_p} \,=\, \omega\big(e_{i_p}, \cdots, e_{i_1}\big)$\, as before,
we may justify that 
$$ a_{i_1\cdots i_p, jk} \,=\, \big((\nabla_{e_k} \nabla\omega)(e_j)\big)(e_{i_p}, \cdots, e_{i_1})\,. $$

Expanding the right hand side of the equation, we have \\[0.1in]
$\begin{array}{rl}
& \ds \big((\nabla_{e_k} \nabla\omega)(e_j)\big)(e_{i_p}, \cdots, e_{i_1}) \\[0.2in]
=& \ds 
	e_k\Big((\nabla_{e_j}\omega)(e_{i_p}, \cdots, e_{i_1})\Big)
	\,-\,(\nabla_{\nabla_{e_k}e_j}\omega) (e_{i_p}, \cdots, e_{i_1}) 
	\,-\, \sum_{r=1}^p (\nabla_{e_j}\omega)\big(e_{i_p}, \cdots, \nabla_{e_k} e_{i_r}, \cdots, e_{i_1}\big) \\[0.2in]
=& \ds 
	e_k\big(a_{i_1\cdots i_p, j}\big) \,-\, a_{i_1 \cdots i_p,s}\,\Gamma_{kj}^s 
	\,-\, \sum_{r, j_r} a_{i_1\cdots j_r \cdots i_p, j} \, \Gamma_{k i_r}^{j_r} \\[0.2in]
=& \ds 
	e_k\big(a_{i_1\cdots i_p, j}\big) \,+\, a_{i_1 \cdots i_p,s}\,\Gamma_{ks}^j
	\,+\, \sum_{r, j_r} a_{i_1\cdots j_r \cdots i_p, j} \, \Gamma_{k j_r}^{i_r} \\[0.2in]
=& \ds 
	a_{i_1\cdots i_p, jk} \,.
\end{array}$ \\
\newpage

Put the above identity to Equation (5). \\[0.1in]
$\begin{array}{lcl}
\Big((\nabla_{e_k} \nabla \omega)(e_j) \,-\, (\nabla_{e_j} \nabla \omega)(e_k)\Big)(e_{i_p}, \cdots, e_{i_1})
&=& \ds a_{i_1\cdots i_p, jk} \,-\, a_{i_1\cdots i_p, kj} \\[0.2in]
% 
\ds \sum_{r=1}^p \omega\Big(e_{i_p}, \cdots, R(e_j,e_k)e_{i_r}, \cdots, e_{i_1}\Big) 
&=& \ds
	\sum_{r, j_r} R_{jk i_r j_r}\, a_{i_1\cdots j_r\cdots i_p} \\ 
\end{array} $\\[0.1in]

Therefore, Equation (5) becomes: 
\begin{equation}
a_{i_1\cdots i_p, jk} \,-\, a_{i_1\cdots i_p, kj} \,=\, \sum_{r, j_r} a_{i_1\cdots j_r\cdots i_p} \, R_{jk i_r j_r}
\end{equation}
when $\omega$ is in the symmetric representation. \\
\vspace*{0.2in}

% % % % % % % % % % % % % % % % % % % % % % % % % % % % % % % % % % %

\begin{center} {\sc 6. Codifferential of differential forms}\end{center}
When $(M,g)$ is an $n$-dimensional manifold, it has a positive volume form
$$ \Omega \,=\, \omega_1 \wedge \omega_2 \wedge \cdots \wedge \omega_n\,. $$ 
The Hodge star operator $\ast$ on $M$ sends any $p$ form to an $(n-p)$ form, 
$$ \ast\big(\omega_{i_p} \wedge \cdots \wedge \omega_{i_1}\big) 
\,=\, \mbox{sgn}(I, I^c) \, \omega_{i_n} \wedge \cdots \wedge \omega_{i_{p+1}}\,. $$

Here we set the $p$-tuple \,$I \,=\, (i_p, \cdots, i_1)$\,. $I^c$ is the complement of $I$, 
defined by \,$(i_{n}, \cdots, i_{p+1})$\, such that the set \,$\big\{i_1, \cdots, i_p, i_{p+1}, \cdots, i_n \big\}$\,
coincides with \,$\big\{1, \cdots, n\big\}$\,. Moreover, we let 
$$ \big(\omega_{i_p} \wedge \cdots \wedge \omega_{i_1}\big) \wedge 
	\big(\omega_{i_n} \wedge \cdots \wedge \omega_{i_{p+1}}\big) \,=\, \mbox{sgn}(I, I^c) \, \Omega\,.$$

As a remark, for every unordered combination of $p$ elements in $\{1, \cdots, n\}$, 
there is a unique ordered permutation $I \,=\, (i_p, i_{p-1}, \cdots, i_1)$ such that 
$i_p < i_{p-1} < \cdots < i_1$. Then, we may specify a complementary tuple of $I$, 
\,$I^c \,=\, \big(i_n, \cdots, i_{p+1}\big)$\,with \,$i_n < i_{n-1} < \cdots < i_{p+1}$\,.
Under this approach, we could specify $I^c$ without ambiguity. \\

The codifferential operator $\delta$ sends a $p$-form to a $(p-1)$-form through
$$ \delta \omega \,=\, (-1)^{n(p+1)+1} \, \ast d \ast \omega \,. $$
Given that \,$\omega \,=\, a_{i_1\cdots i_p}\, \omega_{i_p} \wedge \cdots \wedge \omega_{i_1}$\,, we may prove that 
\begin{equation}
	\delta \omega \,=\, \sum_{r=1}^p (-1)^{r+p^2+1} a_{i_1\cdots i_r\cdots i_p, i_r}\,
	\omega_{i_p} \wedge\cdots\wedge \omega_{i_{r+1}} \wedge \omega_{i_{r-1}} \wedge\cdots\wedge \omega_{i_1} \,.
\end{equation}

For simplicity, we assume that the $p$-tuple \,$I\,=\,(k_p, \cdots, k_1)$\, is fixed, so $\omega$ is defined by
$$ \omega \,=\, a_{k_1\cdots k_p}\, \omega_{k_p}\wedge \cdots \wedge \omega_{k_1}\,.$$
\newpage 

The right hand side of Equation (8) is expanded as follows. \\[0.1in]
$\begin{array}{rl} 
& \ds \sum_{r=1}^p (-1)^{r+p^2+1} \, a_{i_1\cdots i_r\cdots i_p, i_r}\,
	\omega_{i_p} \wedge\cdots\wedge \omega_{i_{r+1}} \wedge \omega_{i_{r-1}} \wedge\cdots\wedge \omega_{i_1} \\[0.3in]
=& \ds 
	\sum_{r, i_r} (-1)^{r+p^2+1} e_{i_r}\big(a_{i_1 \cdots i_p}\big) \,
	\omega_{i_p} \wedge\cdots\wedge \omega_{i_{r+1}} \wedge \omega_{i_{r-1}} \wedge\cdots\wedge \omega_{i_1} \\[0.15in]
& \ds 
	+\, \sum_{r, i_r} (-1)^{r+p^2+1} \, \sum_{s, j_s} \big(a_{i_1 \cdots j_s \cdots i_p}\, \Gamma_{i_r j_s}^{i_s}\big) \, 
	\omega_{i_p} \wedge\cdots\wedge \omega_{i_{r+1}} \wedge \omega_{i_{r-1}} \wedge\cdots\wedge \omega_{i_1}\\[0.3in]
% % % % % % % % % % % % % % % % % % % % % % % % % % % % % % % % % % %
=& \ds 
	\sum_{r=1}^p (-1)^{r+p^2+1}\, e_{k_r}\big(a_{k_1 \cdots k_p}\big) \,
	\omega_{k_p} \wedge\cdots\wedge \widetilde{\omega_{k_r}} \wedge\cdots\wedge \omega_{k_1} \\[0.15in] 
& \ds 
	+\, \sum_{r=1}^p (-1)^{r+p^2+1} \, \sum_{s\neq r} \big(a_{k_1\cdots j_s\cdots k_r\cdots k_p}\, \Gamma_{k_r j_s}^{i_s}\big) \, 
	\omega_{k_p} \wedge\cdots\wedge \widetilde{\omega_{k_r}} \wedge\cdots\wedge  \omega_{i_s} 
	\wedge\cdots\wedge \omega_{k_1} \\[0.15in]	
& \ds 
	+\, \sum_{r=1}^p (-1)^{r+p^2+1} \, \big(a_{k_1\cdots j_r \cdots k_p}\, \Gamma_{i_r j_r}^{i_r}\big)\, 
	\omega_{k_p} \wedge\cdots\wedge \widetilde{\omega_{k_r}} \wedge\cdots\wedge \omega_{k_1} \\[0.3in]
% % % % % % % % % % % % % % % % % % % % % % % % % % % % % % % % % % %
=& \ds 
	\sum_{r=1}^p (-1)^{r+p^2+1}\, e_{k_r}\big(a_{k_1 \cdots k_p}\big) \,
	\omega_{k_p} \wedge\cdots\wedge \widetilde{\omega_{k_r}} \wedge\cdots\wedge \omega_{k_1} \\[0.15in] 
& \ds 
	+\, \sum_{(r,s):\, r\neq s} (-1)^{r+p^2+1} \, a_{k_1 \cdots k_s \cdots k_r \cdots k_p}\, \Gamma_{k_r k_s}^{i_s} \,
	\omega_{k_p} \wedge\cdots\wedge \widetilde{\omega_{k_r}} \wedge\cdots\wedge  \omega_{i_s} 
	\wedge\cdots\wedge \omega_{k_1} \\[0.15in]	
& \ds 
	+\, \sum_{r=1}^p (-1)^{r+p^2+1} \, a_{k_1 \cdots k_p} \, \Gamma_{i_r k_r}^{i_r} \,
	\omega_{k_p} \wedge\cdots\wedge \widetilde{\omega_{k_r}} \wedge\cdots\wedge \omega_{k_1} \\[0.3in]
% % % % % % % % % % % % % % % % % % % % % % % % % % % % % % % % % % %
=& \ds 
	\sum_{r=1}^p (-1)^{r+p^2+1}\, e_{k_r}\big(a_{k_1 \cdots k_p}\big) \,
	\omega_{k_p} \wedge\cdots\wedge \widetilde{\omega_{k_r}} \wedge\cdots\wedge \omega_{k_1} \\[0.15in] 
& \ds 	
	+\, \sum_{r\neq s} (-1)^{r+p^2+1} \, a_{k_1\cdots k_p} \, \Gamma_{k_r k_s}^{k_r} \, (-1)^{r-s+1}\, 
	\omega_{k_p} \wedge\cdots\wedge \widetilde{\omega_{k_s}} \wedge\cdots\wedge \omega_{k_1} \\[0.15in]	
& \ds 
	+\, \sum_{r\neq s} (-1)^{r+p^2+1} \, a_{k_1 \cdots k_p}\, \Gamma_{k_r k_s}^{k_{\alpha(s)}} \,
	\omega_{k_p} \wedge\cdots\wedge \widetilde{\omega_{k_r}} \wedge\cdots\wedge  \omega_{k_{\alpha(s)}} 
	\wedge\cdots\wedge \omega_{k_1} \\[0.15in]	
& \ds 
	+\, \sum_{r=1}^p (-1)^{r+p^2+1} \, a_{k_1 \cdots k_p} \, \Gamma_{i_r k_r}^{i_r} \,
	\omega_{k_p} \wedge\cdots\wedge \widetilde{\omega_{k_r}} \wedge\cdots\wedge \omega_{k_1} \\[0.3in]

% % % % % % % % % % % % % % % % % % % % % % % % % % % % % % % % % % %
=& \ds 
	\sum_{r=1}^p (-1)^{r+p^2+1}\, e_{k_r}\big(a_{k_1 \cdots k_p}\big) \,
	\omega_{k_p} \wedge\cdots\wedge \widetilde{\omega_{k_r}} \wedge\cdots\wedge \omega_{k_1} \\[0.15in] 
& \ds 
	+\, \sum_{r\neq s} (-1)^{r+p^2+1} \, a_{k_1 \cdots k_p}\, \Gamma_{k_r k_s}^{k_{\alpha(s)}} \,
	\omega_{k_p} \wedge\cdots\wedge \widetilde{\omega_{k_r}} \wedge\cdots\wedge  \omega_{k_{\alpha(s)}} 
	\wedge\cdots\wedge \omega_{k_1} \\[0.15in]	
& \ds 
	+\, \sum_{s=1}^p (-1)^{p^2+s}\, a_{k_1\cdots k_p} \, \Gamma_{k_r k_s}^{k_r} \,
	\omega_{k_p} \wedge\cdots\wedge \widetilde{\omega_{k_s}} \wedge\cdots\wedge \omega_{k_1} \\[0.15in]	
& \ds 
	+\, \sum_{r=1}^p (-1)^{r+p^2+1} \, a_{k_1 \cdots k_p} \, \Gamma_{i_r k_r}^{i_r} \,
	\omega_{k_p} \wedge\cdots\wedge \widetilde{\omega_{k_r}} \wedge\cdots\wedge \omega_{k_1} \\
\end{array} $ \\

\newpage 
% % % % % % % % % % % % % % % % % % % % % % % % % % % % % % % % % % %
Therefore, the right hand side becomes:\\[0.1in]
$\begin{array}{rl} 
= & \ds 
	\sum_{r=1}^p (-1)^{r+p^2+1}\, e_{k_r}\big(a_{k_1 \cdots k_p}\big) \,
	\omega_{k_p} \wedge\cdots\wedge \widetilde{\omega_{k_r}} \wedge\cdots\wedge \omega_{k_1} \\[0.15in] 
& \ds 
	+\, \sum_{r\neq s} (-1)^{r+p^2+1} \, a_{k_1 \cdots k_p}\, \Gamma_{k_r k_s}^{k_{\alpha}} \,
	\omega_{k_p} \wedge\cdots\wedge \widetilde{\omega_{k_r}} \wedge\cdots\wedge  \omega_{k_{\alpha}}^{(s)}
	\wedge\cdots\wedge \omega_{k_1} \\[0.15in]	
& \ds 
	+\, \sum_{r=1}^p (-1)^{r+p^2+1} \, a_{k_1 \cdots k_p} \, \Gamma_{k_{\alpha} k_r}^{k_{\alpha}} \,
	\omega_{k_p} \wedge\cdots\wedge \widetilde{\omega_{k_r}} \wedge\cdots\wedge \omega_{k_1} \,.
\end{array}$ \\[0.1in]

In the second term above, $\omega_{k_{\alpha}}^{(s)}$ means that the term $\omega_{k_{\alpha}}$ 
lies in the $s$-th position of the wedge product. 
In order to justify Equation (8), we specify the complement of the $p$-tuple $I$ as
$$ I^c \,=\, \big(k_n, k_{n-1}, \cdots, k_{p+1}\big)\,. $$
Immediately, we have
$$  \ast \, \omega \,=\, a_{k_1\cdots k_p}\, \mbox{sgn}(I, I^c)\, \omega_{k_n} \wedge\cdots\wedge \omega_{k_{p+1}}\,.  $$
% % % % %
Therefore, \\[0.1in]
$\begin{array}{rcl} 
d\, \ast\, \omega &=& \ds 
	\mbox{sgn}(I, I^c)\, da_{k_1\cdots k_p} \wedge \omega_{k_n} \wedge\cdots\wedge \omega_{k_{p+1}} \\[0.1in]
&&\ds
	+\, \sum_{\alpha=p+1}^n a_{k_1\cdots k_p} \mbox{sgn}(I, I^c)\, (-1)^{n-\alpha}\, 
	\omega_{k_n} \wedge\cdots\wedge d\omega_{k_\alpha} \wedge\cdots\wedge \omega_{k_{p+1}} \\[0.2in]
% % % % % % % % % % % % % % % % % % % % % % % % % % % % % % % % % % %
&=& \ds 
	\sum_{r=1}^p \mbox{sgn}(I, I^c)\, e_{k_r}\big(a_{k_1\cdots k_p}\big) \,
	\omega_{k_r} \wedge \omega_{k_n} \wedge\cdots\wedge \omega_{k_{p+1}} \\[0.1in]
&& \ds 
	+\, \sum_{\alpha=p+1}^n (-1)^{n-\alpha}\, a_{k_1\cdots k_p} \mbox{sgn}(I, I^c)\, \omega_{k_\alpha j}\wedge \omega_j 
	\wedge \omega_{k_n} \wedge\cdots\wedge \widetilde{\omega_{k_\alpha}} \wedge\cdots\wedge \omega_{k_{p+1}} \\[0.2in]
% % % % % % % % % % % % % % % % % % % % % % % % % % % % % % % % % % %
&=& \ds 
	\sum_{r=1}^p \mbox{sgn}(I, I^c)\, e_{k_r}\big(a_{k_1\cdots k_p}\big) \,
	\omega_{k_r} \wedge \omega_{k_n} \wedge\cdots\wedge \omega_{k_{p+1}} \\[0.1in]
&& \ds 
	+\, \sum_{\alpha=p+1}^n (-1)^{n-\alpha}\, a_{k_1\cdots k_p} \mbox{sgn}(I, I^c)\, \Gamma_{m\, k_\alpha}^j \, \omega_m \wedge \omega_j 
	\wedge \omega_{k_n} \wedge\cdots\wedge \widetilde{\omega_{k_\alpha}} \wedge\cdots\wedge \omega_{k_{p+1}} \\[0.2in]
% % % % % % % % % % % % % % % % % % % % % % % % % % % % % % % % % % %
&=& \ds 
	\sum_{r=1}^p \mbox{sgn}(I, I^c)\, e_{k_r}\big(a_{k_1\cdots k_p}\big) \,
	\omega_{k_r} \wedge \omega_{k_n} \wedge\cdots\wedge \omega_{k_{p+1}} \\[0.1in] 
&& \ds 
	+\, \sum_{\alpha} (-1)^{n-\alpha}\, a_{k_1\cdots k_p} \mbox{sgn}(I, I^c)\, \Gamma_{k_\alpha \, k_\alpha}^j \,
	\omega_{k_\alpha} \wedge \omega_j \wedge \omega_{k_n} \wedge\cdots\wedge \widetilde{\omega_{k_\alpha}} 
	\wedge\cdots\wedge \omega_{k_{p+1}} \\[0.1in]
&& \ds 
	+\, \sum_{s=1}^p (-1)^{n-\alpha}\, a_{k_1\cdots k_p} \mbox{sgn}(I, I^c)\, \Gamma_{k_s\, k_\alpha}^j \,
 	\omega_{k_s} \wedge  \omega_j \wedge \omega_{k_n} \wedge\cdots\wedge \widetilde{\omega_{k_\alpha}} 
	\wedge\cdots\wedge \omega_{k_{p+1}} \\[0.2in]
% % % % % % % % % % % % % % % % % % % % % % % % % % % % % % % % % % %
&=& \ds 
\sum_{r=1}^p \mbox{sgn}(I, I^c)\, e_{k_r}\big(a_{k_1\cdots k_p}\big) \,
	\omega_{k_r} \wedge \omega_{k_n} \wedge\cdots\wedge \omega_{k_{p+1}} \\[0.1in] 
&& \ds 
	-\, \sum_{r, \alpha} a_{k_1\cdots k_p} \mbox{sgn}(I, I^c)\, \Gamma_{k_\alpha \, k_\alpha}^{k_r} \,
	\omega_{k_r} \wedge \omega_{k_n} \wedge\cdots\wedge \omega_{k_{p+1}} \\[0.1in]
&& \ds 
	+\, \sum_{s\neq r} a_{k_1\cdots k_p} \mbox{sgn}(I, I^c)\, (-1)^{n-\alpha}\, \Gamma_{k_s\, k_\alpha}^{k_r} \,
 	\omega_{k_s} \wedge  \omega_{k_r} \wedge \omega_{k_n} \wedge\cdots\wedge \widetilde{\omega_{k_\alpha}} 
	\wedge\cdots\wedge \omega_{k_{p+1}} \,.
\end{array}$
\newpage

% % % % % % % % % % % % % % % % % % % % % % % % % % % % % % % % % % %
We then compute for \,$(-1)^{n(p+1)+1} \, \ast d \ast \omega $.\, First of all,  \\[0.1in]
$\begin{array}{rl}
&\ds 
	(-1)^{n(p+1)+1}\, \ast\,\bigg(\sum_{r=1}^p \mbox{sgn}(I, I^c)\, e_{k_r}\big(a_{k_1\cdots k_p}\big) \,
	\omega_{k_r} \wedge \omega_{k_n} \wedge\cdots\wedge \omega_{k_{p+1}} \bigg) \\[0.3in]
=& \ds 
	\sum_r \left[\begin{array}{l}
		(-1)^{n(p+1)+1}\, \mbox{sgn}(I, I^c)\, e_{k_r}\big(a_{k_1\cdots k_p}\big) \, 
		\mbox{sgn}\Big(\big(k_r, k_n, \cdots, k_{p+1}\big)\,,\, \big(k_p, \cdots, \widetilde{k_r}, \cdots, k_1\big)\Big) \\
		\cdot \, \Big(\omega_{k_p} \wedge\cdots\wedge \widetilde{\omega_{k_r}} \wedge\cdots\wedge \omega_{k_1}\Big) 
	\end{array}\right]\\[0.3in]
=& \ds 
	\sum_r (-1)^{np+n+1}\, (-1)^{n-r}\, \mbox{sgn}(I, I^c)\, \mbox{sgn}(I^c, I)\, e_{k_r}\big(a_{k_1\cdots k_p}\big) \, 
	\omega_{k_p} \wedge\cdots\wedge \widetilde{\omega_{k_r}} \wedge\cdots\wedge \omega_{k_1} \\[0.3in]
=& \ds 
	\sum_r (-1)^{np-r+1} \, (-1)^{(n-p)p} \, e_{k_r}\big(a_{k_1\cdots k_p}\big) \, 
	\omega_{k_p} \wedge\cdots\wedge \widetilde{\omega_{k_r}} \wedge\cdots\wedge \omega_{k_1} \\[0.3in]
=& \ds 
	\sum_r (-1)^{p^2+r+1} \,  e_{k_r}\big(a_{k_1\cdots k_p}\big) \, 
	\omega_{k_p} \wedge\cdots\wedge \widetilde{\omega_{k_r}} \wedge\cdots\wedge \omega_{k_1} \\
\end{array}$ \\[0.1in]

Apply the Hodge star operator to the second term of \,$d\, \ast \,\omega$. \\[0.1in]
$\begin{array}{rl}
&\ds 
	(-1)^{n(p+1)+1}\, \ast\,\bigg( 
	-\, \sum_{r} a_{k_1\cdots k_p} \, \mbox{sgn}(I, I^c)\, \Gamma_{k_\alpha \, k_\alpha}^{k_r} \,
	\omega_{k_r} \wedge \omega_{k_n} \wedge\cdots\wedge \omega_{k_{p+1}} \bigg) \\[0.3in]
=& \ds 
	\sum_{r} \left[\begin{array}{l}
		(-1)^{np+n} \, a_{k_1\cdots k_p} \, \mbox{sgn}(I, I^c)\, \Gamma_{k_\alpha \, k_\alpha}^{k_r} \,
		\mbox{sgn}\Big(\big(k_r, k_n, \cdots, k_{p+1}\big)\,,\,\big(k_p, \cdots, \widetilde{k_r}, \cdots, k_1\big)\Big) \\
		\cdot \, \Big(\omega_{k_p} \wedge\cdots\wedge \widetilde{\omega_{k_r}} \wedge\cdots\wedge \omega_{k_1}\Big) 
	\end{array}\right] \\[0.3in]
=& \ds 
	\sum_{r} (-1)^{np+n}\, (-1)^{n-r}\, a_{k_1\cdots k_p} \, \Gamma_{k_\alpha \, k_\alpha}^{k_r} \,
	\mbox{sgn}(I, I^c)\, \mbox{sgn}(I^c, I) \,
	\omega_{k_p} \wedge\cdots\wedge \widetilde{\omega_{k_r}} \wedge\cdots\wedge \omega_{k_1} \\[0.3in]
=& \ds 
	\sum_{r} (-1)^{np-r}\, (-1)^{(n-p)p}\, a_{k_1\cdots k_p} \, \Gamma_{k_\alpha \, k_\alpha}^{k_r} \,
	\omega_{k_p} \wedge\cdots\wedge \widetilde{\omega_{k_r}} \wedge\cdots\wedge \omega_{k_1} \\[0.3in]
=& \ds 
	\sum_r (-1)^{p^2+r+1} \, a_{k_1\cdots k_p} \, \Gamma_{k_\alpha \, k_r}^{k_\alpha} \,
	\omega_{k_p} \wedge\cdots\wedge \widetilde{\omega_{k_r}} \wedge\cdots\wedge \omega_{k_1} \\
\end{array}$ \\[0.1in]

We compute for the third term of \,$d\, \ast \,\omega$. .\\[0.1in]
$\begin{array}{rl}
&\ds 
	(-1)^{n(p+1)+1}\,\ast\, \bigg( 
	\sum_{s\neq r} a_{k_1\cdots k_p} \mbox{sgn}(I, I^c)\, (-1)^{n-\alpha}\, \Gamma_{k_s\, k_\alpha}^{k_r} \,
 	\omega_{k_s} \wedge  \omega_{k_r} \wedge \omega_{k_n} \wedge\cdots\wedge \widetilde{\omega_{k_\alpha}} 
	\wedge\cdots\wedge \omega_{k_{p+1}}\bigg)\\[0.2in]
=& \ds 
	\sum_{s\neq r} \left[\begin{array}{l}
	(-1)^{np+n+1}\, a_{k_1\cdots k_p} \mbox{sgn}(I, I^c)\, (-1)^{n-\alpha}\, \Gamma_{k_s\, k_\alpha}^{k_r} \\[2pt]
	\cdot \, \mbox{sgn}\Big(\big(k_s, k_r, k_n, \cdots, \widetilde{k_\alpha}, \cdots, k_{p+1}\big)\,,\,
	\big(k_\alpha, k_p, \cdots, \widetilde{k_r}, \cdots, \widetilde{k_s}, \cdots, k_1\big)\Big) \\[2pt]
	\cdot \, \Big(\omega_{k_\alpha} \wedge \omega_{k_p} \wedge\cdots\wedge \widetilde{\omega_{k_r}} \wedge\cdots\wedge \widetilde{\omega_{k_s}} 
	\wedge\cdots\wedge \omega_1 \Big)
	\end{array}\right] \\[0.5in]
=& \ds 
	\sum_{s\neq r} \left[\begin{array}{l}
	(-1)^{np+\alpha+1}\, a_{k_1\cdots k_p} \mbox{sgn}(I, I^c)\, \Gamma_{k_s\, k_\alpha}^{k_r} \\[2pt]
	\cdot \,\mbox{sgn}\Big(\big(k_s, k_r, k_n, \cdots, \widetilde{k_\alpha}, \cdots, k_{p+1}\big)\,,\,
	\big(k_\alpha, k_p, \cdots, \widetilde{k_r}, \cdots, \widetilde{k_s}, \cdots, k_1\big)\Big) \\[2pt]
	\cdot \, \Big(\omega_{k_\alpha} \wedge \omega_{k_p} \wedge\cdots\wedge \widetilde{\omega_{k_r}} \wedge\cdots\wedge \widetilde{\omega_{k_s}} 
	\wedge\cdots\wedge \omega_1 \Big)
	\end{array} \right]
\end{array}$
\newpage 

When $s>r$, \\[0.1in]
$\begin{array}{rl} 
& \ds  
	\mbox{sgn}\Big(\big(k_s, k_r, k_n, \cdots, \widetilde{k_\alpha}, \cdots, k_{p+1}\big)\,,\,
	\big(k_\alpha, k_p, \cdots, \widetilde{k_s}, \cdots, \widetilde{k_r}, \cdots, k_1\big)\Big)\\[4pt]
& \ds 
	\cdot \, \Big(\omega_{k_\alpha} \wedge \omega_{k_p} \wedge\cdots\wedge \widetilde{\omega_{k_s}} 
	\wedge\cdots\wedge \widetilde{\omega_{k_r}} \wedge\cdots\wedge \omega_1\Big) \\[0.2in]
=& \ds 
	(-1)^{\alpha-p-1} \, (-1)^{n-s+1}\, (-1)^{n-r}\, \mbox{sgn}(I^c, I)  \,
	(-1)^{p-r-1}\, \omega_{k_p} \wedge\cdots\wedge \widetilde{\omega_{k_s}} \wedge\cdots\wedge 
	\omega_{k_\alpha}^{(r)} \wedge\cdots\wedge \omega_1 \\[0.2in]
=& \ds 
	(-1)^{\alpha+s+1} \, \mbox{sgn}(I^c, I)\, 
	\omega_{k_p} \wedge\cdots\wedge \widetilde{\omega_{k_s}} \wedge\cdots\wedge 
	\omega_{k_\alpha}^{(r)} \wedge\cdots\wedge \omega_1
\end{array}$ \\[0.1in]

When $s<r$, \\[0.1in]
$\begin{array}{rl} 
& \ds 
	\mbox{sgn}\Big(\big(k_s, k_r, k_n, \cdots, \widetilde{k_\alpha}, \cdots, k_{p+1}\big)\,,\,
	\big(k_\alpha, k_p, \cdots, \widetilde{k_r}, \cdots, \widetilde{k_s}, \cdots, k_1\big)\Big)\\[4pt]
& \ds 
	\cdot\, \Big(\omega_{k_\alpha} \wedge \omega_{k_p} \wedge\cdots\wedge \widetilde{\omega_{k_r}} 
	\wedge\cdots\wedge \widetilde{\omega_{k_s}} \wedge\cdots\wedge \omega_1\Big) \\[0.2in]

=& \ds (-1)^{\alpha-p-1}\, (-1)^{n-r}\, (-1)^{n-s} \, \mbox{sgn}(I^c, I)\, (-1)^{p-r}
	\omega_{k_p} \wedge\cdots\wedge \omega_{k_\alpha}^{(r)} \wedge\cdots\wedge \widetilde{\omega_{k_s}} 
	\wedge\cdots\wedge \omega_1 \\[0.2in]
=& \ds 
	(-1)^{\alpha+s+1} \, \mbox{sgn}(I^c, I)\, 
	\omega_{k_p} \wedge\cdots\wedge \omega_{k_\alpha}^{(r)} \wedge\cdots\wedge \widetilde{\omega_{k_s}} 
	\wedge\cdots\wedge \omega_1 \,.\\[0.2in]
\end{array}$ \\[0.1in]

Therefore, the third term becomes \\[0.1in]
$\begin{array}{rl} 
& \ds \sum_{s\neq r} \left[ \begin{array}{l}
		(-1)^{np+\alpha+1}\, a_{k_1\cdots k_p} \mbox{sgn}(I, I^c)\, \Gamma_{k_s\, k_\alpha}^{k_r} \\[4pt]
		\cdot \,(-1)^{\alpha+s+1} \, \mbox{sgn}(I^c, I) \, 
		\Big( \omega_{k_p} \wedge\cdots\wedge \omega_{k_\alpha}^{(r)} \wedge\cdots\wedge \widetilde{\omega_{k_s}} 
		\wedge\cdots\wedge \omega_1 \Big) 
	\end{array} \right] \\[0.4in]
=& \ds 
	\sum_{s\neq r} (-1)^{np+\alpha+1}\, (-1)^{\alpha+s+1}\, (-1)^{(n-p)p} \,
	a_{k_1\cdots k_p} \, \Gamma_{k_s\, k_\alpha}^{k_r} \, 
	\omega_{k_p} \wedge\cdots\wedge \omega_{k_\alpha}^{(r)} \wedge\cdots\wedge \widetilde{\omega_{k_s}} \wedge\cdots\wedge \omega_1 
	\\[0.3in]
=& \ds 
	\sum_{s\neq r} (-1)^{p^2+s}\, a_{k_1\cdots k_p} \, \Gamma_{k_s\, k_\alpha}^{k_r} \, 
	\omega_{k_p} \wedge\cdots\wedge \omega_{k_\alpha}^{(r)} \wedge\cdots\wedge \widetilde{\omega_{k_s}} \wedge\cdots\wedge \omega_1 \\[0.3in]
=& \ds 
	\sum_{r\neq s} (-1)^{p^2+r +1} \, a_{k_1\cdots k_p} \, \Gamma_{k_r\, k_s}^{k_\alpha} \,
	\omega_{k_p} \wedge\cdots\wedge \omega_{k_\alpha}^{(s)} \wedge\cdots\wedge \widetilde{\omega_{k_r}} \wedge\cdots\wedge \omega_1\,. \\
\end{array}$ \\[0.1in]

By comparsion of components on both sides, we have established Equation (8). \\
\newpage
% % % % % % % % % % % % % % % % % % % % % % % % % % % % % % % % % % %

\begin{center} {\sc 7. Laplacian of differential forms}\end{center}
The Laplacian of a $p$-form is defined by
$$ \Delta \omega \,=\, -\,d \, \delta \omega \,-\, \delta\, d\, \omega\,. $$
We first consider the case that $\omega$ is a 1-form. Let \,$\omega=\sum_{j} a_j \, \omega_j$\,.
In terms of its covariant derivative, we have \,$d\omega \,=\, \sum_j a_{j,k} \, \omega_k \wedge \omega_j$\,.
By Equation (8), we find that 
$$ \delta \omega \,=\, \delta\big( a_j\, \omega_j \big) \,=\, -\, \sum_j a_{j,j} 
\quad \implies \quad d\, \delta \, \omega \,=\, -\, \sum_j da_{j,j}\,. $$
The second covariant derivative of $\omega$ is described by
$$ a_{i,jk}\, \omega_k \,=\, da_{i,j} \,+\, a_{i,s}\, \omega_{sj} \,+\, a_{r,j}\, \omega_{ri}\,.$$
Note that 
$$ \sum_{j,k} a_{j,jk} \, \omega_k 
	\,=\, \sum_{j} da_{j,j} \,+\, \sum_{j,s} a_{j,s}\, \omega_{sj} \,+\, \sum_{r, j} a_{r, j}\, \omega_{rj} 
	\,=\, \sum_{j} da_{j,j}\,.$$
Therefore,
$$d\, \delta \, \omega \,=\,  - a_{j,jk}\, \omega_k\,.$$
\vspace*{0.1in}

On the other hand, we have \, $\delta\, d\, \omega \,=\, \delta\big( a_{j,k}\,\omega_k \wedge \omega_j \big)$\,.
Let \,$b_{jk}\,=\, a_{j,k}$\, and consider \,$\beta \,=\, b_{jk}\,\omega_k \wedge \omega_j$\, in the following.
By Equation (8), \\[0.1in]
$\begin{array}{rcl}
\delta\, d\, \omega &=& \ds \sum_{\alpha=1,2} (-1)^{\alpha +5} \, b_{i_1\cdots i_\alpha\cdots i_2, i_\alpha}\, 
	\omega_{i_2} \wedge\cdots\wedge \widetilde{\omega_{i_\alpha}} \wedge\cdots\wedge \omega_{i_1} \\[0.2in]
&=& \ds 
	(-1)^6\, b_{j_1 i_2, j_1}\, \omega_{i_2} \,+\, (-1)^7 \, b_{i_1 j_2, j_2}\, \omega_{i_1} \\[0.1in]
&=& \ds 
	b_{jk, j}\, \omega_k \,-\, b_{kj, j}\, \omega_k \\[0.1in] 
&=& \ds 
	\big( b_{jk, j} \,-\, b_{kj, j}\big) \, \omega_k
\end{array}$ \\[0.1in]

The first covariant derivatives of $\beta$ are found by
$$ b_{jk,l}\, \omega_l \,=\, db_{jk} \,+\, b_{rk}\, \omega_{rj} \,+\, b_{js}\, \omega_{sk}
\,=\, da_{j,k} \,+\, a_{r,k}\, \omega_{rj} \,+\, a_{j,s}\, \omega_{sk}\,.$$
So, \,$ b_{jk,l} \,=\, e_l(a_{j,k}) \,+\, a_{r,k}\, \Gamma_{lr}^j \,+\, a_{j,s}\, \Gamma_{ls}^k$\,.
Note that we also have
$$ a_{j, kl}\, \omega_l \,=\, da_{j,k} \,+\, a_{j,s}\, \omega_{sk} \,+\, a_{r,k}\, \omega_{rj}\,. $$
Therefore, \,$b_{jk,l} \,=\, a_{j,kl}$\, for every $j$, $k$, $l$. It also means that
$$ \delta\, d\, \omega \,=\, \big( a_{j,kj} \,-\, a_{k,jj} \big)\, \omega_k $$
\newpage 

Adding up the above terms, \\[0.1in]
$\begin{array}{rcl} 
\Delta\,\omega &=& \ds 
	a_{j,jk}\, \omega_k \,-\, a_{j,kj}\, \omega_k \,+\, a_{k,jj}\, \omega_k \\[0.1in]
&=& \ds 
	a_{k,jj}\,\omega_k \,+\, \big(a_{j,jk} \,-\, a_{j, kj}\big)\, \omega_k \\[0.1in]
&=& \ds 
	a_{k,jj} \,\omega_k \,+\, a_r \, R_{jkjr}\, \omega_k \\[0.1in]
&=& \ds 
	\big( a_{k,jj} \,-\, a_r R_{rk} \big) \,\omega_k \,.
\end{array}$\\[0.1in]

Here $R_{rk} \,=\, \sum_j R_{rjjk}$ is a coefficient of the Ricci tensor. 
We may let
$$ \nabla^{\ast} \nabla\,\omega \,=\, \sum_{j,k} a_{k,jj}\, \omega_k \man
E(\omega) \,=\, \sum a_r\,R_{rk}\, \omega_k\,.$$ 
% % % % % % % % % % % % % % % % % % % % % % % % % % % % % % % % % % %
As a result, \,$\Delta\,\omega \,=\, \nabla^\ast \nabla\,\omega \,-\, E(\omega)$\,.
We are going to find $\Delta\,\omega$ and prove this result to a general $p$-form on the manifold $M$. 
Let \,$\omega \,=\, a_{i_1\cdots i_p}\, \omega_{i_p}\wedge\cdots\wedge\omega_{i_1}$\,. \\
$$ \delta\,\omega \,=\,
	\sum_{r=1}^p (-1)^{p^2+r+1}\, a_{i_1\cdots i_r\cdots i_p,i_r}\, \omega_{i_p}\wedge\cdots\wedge \widetilde{\omega_{i_r}}
	\wedge\cdots\wedge \omega_{i_1}. $$

It leads to \\[4pt]
$\begin{array}{rcl}
d\, \delta\, \omega &=& \ds 
	\sum_r (-1)^{p^2+r+1}\, da_{i_1\cdots i_r\cdots i_p,i_r} \wedge 
	\omega_{i_p}\wedge\cdots\wedge \widetilde{\omega_{i_r}} \wedge\cdots\wedge \omega_{i_1} \\[4pt]
&& \ds 
	+\, \sum_{s>r} (-1)^{p^2+r+1}\, a_{i_1\cdots i_r\cdots i_p,i_r} \, (-1)^{p-s}\,
	\omega_{i_p}\wedge\cdots\wedge d\omega_{i_s} \wedge\cdots\wedge \widetilde{\omega_{i_r}} 
	\wedge\cdots\wedge \omega_{i_1} \\[4pt]
&& \ds 
	+\, \sum_{s<r} (-1)^{p^2+r+1}\, a_{i_1\cdots i_r\cdots i_p,i_r} \, (-1)^{p-s-1}\,
	\omega_{i_p}\wedge\cdots\wedge \widetilde{\omega_{i_r}} \wedge\cdots\wedge d\omega_{i_s} 
	\wedge\cdots\wedge \omega_{i_1} \\[0.2in]
&=& \ds 
	\sum_r (-1)^{p^2+r+1}\, da_{i_1\cdots i_r\cdots i_p,i_r} \wedge 
	\omega_{i_p}\wedge\cdots\wedge \widetilde{\omega_{i_r}} \wedge\cdots\wedge \omega_{i_1} \\[4pt]
&& \ds 
	+\, \sum_{s>r} (-1)^{r+s+1}\, a_{i_1\cdots i_r\cdots i_p,i_r} \,
	\omega_{i_s j}\wedge \omega_j \wedge \omega_{i_p} \wedge\cdots\wedge \widetilde{\omega_{i_s}} \wedge\cdots\wedge 
	\widetilde{\omega_{i_r}} \wedge\cdots\wedge \omega_{i_1} \\[4pt]
&& \ds 
	+\, \sum_{s<r} (-1)^{r+s}\, a_{i_1\cdots i_r\cdots i_p,i_r} \,
	\omega_{i_s j}\wedge \omega_j  \wedge \omega_{i_p}\wedge\cdots\wedge \widetilde{\omega_{i_r}} 
	\wedge\cdots\wedge \widetilde{\omega_{i_s}} \wedge\cdots\wedge \omega_{i_1} \\[0.2in]
% % % % % % % % % % % % % % % % % % % % % % % % % % % % % % % % % % %
&=& \ds 
	\sum_r (-1)^{p^2+r+1}\, \left[\begin{array}{l}
		\Big(a_{i_1\cdots i_r\cdots i_p, i_rk}\,\omega_k \,-\, a_{i_1\cdots i_r \cdots i_p,s}\, \omega_{si_r} 
		\,-\, \sum_{l, j_l}a_{i_1\cdots j_l\cdots i_p,i_r}\, \omega_{j_l i_l}\Big) \\[4pt]
		\wedge \omega_{i_p}\wedge\cdots\wedge \widetilde{\omega_{i_r}} \wedge\cdots\wedge \omega_{i_1} \\ 
	\end{array} \right] \\[4pt]
&& \ds 
	+\, \sum_{s>r} (-1)^{r+s+1}\, a_{i_1\cdots i_r\cdots i_p,i_r} \,
	\omega_{i_s j}\wedge \omega_j \wedge \omega_{i_p} \wedge\cdots\wedge \widetilde{\omega_{i_s}} \wedge\cdots\wedge 
	\widetilde{\omega_{i_r}} \wedge\cdots\wedge \omega_{i_1} \\[4pt]
&& \ds 
	+\, \sum_{s<r} (-1)^{r+s}\, a_{i_1\cdots i_r\cdots i_p,i_r} \,
	\omega_{i_s j}\wedge \omega_j  \wedge \omega_{i_p}\wedge\cdots\wedge \widetilde{\omega_{i_r}} 
	\wedge\cdots\wedge \widetilde{\omega_{i_s}} \wedge\cdots\wedge \omega_{i_1} \\[0.3in]
% % % % % % % % % % % % % % % % % % % % % % % % % % % % % % % % % % %
&=& \ds 
	\sum_r (-1)^{p^2+r+1}\, a_{i_1\cdots i_r\cdots i_p, i_rk}\,
	\omega_k \wedge \omega_{i_p}\wedge\cdots\wedge \widetilde{\omega_{i_r}} \wedge\cdots\wedge \omega_{i_1} \\[4pt]
&& \ds 
	-\, \sum_r (-1)^{p^2+r+1}\, a_{i_1\cdots i_r \cdots i_p,s}\, \omega_{si_r} \,
	\wedge \omega_{i_p}\wedge\cdots\wedge \widetilde{\omega_{i_r}} \wedge\cdots\wedge \omega_{i_1} \\[4pt]
&& \ds 
	-\, \sum_r \sum_{l, j_l} (-1)^{p^2+r+1}\,  a_{i_1\cdots j_l\cdots i_p,i_r}\, \omega_{j_l i_l} \,
	\wedge \omega_{i_p}\wedge\cdots\wedge \widetilde{\omega_{i_r}} \wedge\cdots\wedge \omega_{i_1} \\[4pt]
&& \ds 
	+\, \sum_{s>r} (-1)^{r+s+1}\, a_{i_1\cdots i_r\cdots i_p,i_r} \,
	\omega_{i_s j}\wedge \omega_j \wedge \omega_{i_p} \wedge\cdots\wedge \widetilde{\omega_{i_s}} \wedge\cdots\wedge 
	\widetilde{\omega_{i_r}} \wedge\cdots\wedge \omega_{i_1} \\[4pt]
&& \ds 
	+\, \sum_{s<r} (-1)^{r+s}\, a_{i_1\cdots i_r\cdots i_p,i_r} \,
	\omega_{i_s j}\wedge \omega_j  \wedge \omega_{i_p}\wedge\cdots\wedge \widetilde{\omega_{i_r}} 
	\wedge\cdots\wedge \widetilde{\omega_{i_s}} \wedge\cdots\wedge \omega_{i_1}
\end{array}$
\newpage
% % % % % % % % % % % % % % % % % % % % % % % % % % % % % % % % % % %
$\begin{array}{rcl}
d\, \delta\, \omega &=& \ds 
	\sum_r (-1)^{p^2+r+1}\, a_{i_1\cdots i_r\cdots i_p, i_rk}\,
	\omega_k \wedge \omega_{i_p}\wedge\cdots\wedge \widetilde{\omega_{i_r}} \wedge\cdots\wedge \omega_{i_1} \\[0.1in]
&& \ds 
	+\, \sum_r (-1)^{p^2+r}\, a_{i_1\cdots i_r \cdots i_p,s}\, \omega_{si_r} \,
	\wedge \omega_{i_p}\wedge\cdots\wedge \widetilde{\omega_{i_r}} \wedge\cdots\wedge \omega_{i_1} \\[0.1in]
&& \ds 
	+\, \sum_{l>r} (-1)^{p^2+r}\,  a_{i_1\cdots i_r \cdots  j_l\cdots  i_p,i_r}\, \omega_{j_l i_l} \,
	\wedge \omega_{i_p}\wedge\cdots\wedge \widetilde{\omega_{i_r}} \wedge\cdots\wedge \omega_{i_1} \\[0.1in]
&& \ds 
	+\, \sum_{l<r} (-1)^{p^2+r}\,  a_{i_1 \cdots j_l \cdots i_r \cdots i_p,i_r}\, \omega_{j_l i_l} \,
	\wedge \omega_{i_p}\wedge\cdots\wedge \widetilde{\omega_{i_r}} \wedge\cdots\wedge \omega_{i_1} \\[0.1in]
&& \ds 
	+\, \sum_r (-1)^{p^2+r}\,  a_{i_1 \cdots j_r \cdots i_p,i_r}\, \omega_{j_r i_r} \,
	\wedge \omega_{i_p}\wedge\cdots\wedge \widetilde{\omega_{i_r}} \wedge\cdots\wedge \omega_{i_1} \\[0.1in]
&& \ds 
	+\, \sum_{s>r} (-1)^{r+s+1}\, a_{i_1\cdots i_r \cdots j_s \cdots i_p,i_r} \, (-1)^{p-s}\,
	\omega_{j_s i_s}  \wedge \omega_{i_p} \wedge\cdots\wedge \widetilde{\omega_{i_r}} \wedge\cdots\wedge \omega_{i_1} \\[0.1in]
&& \ds 
	+\, \sum_{s<r} (-1)^{r+s}\, a_{i_1\cdots j_s \cdots  i_r\cdots i_p,i_r} \, (-1)^{p-s-1}\, 
	\omega_{j_s i_s}  \wedge \omega_{i_p} \wedge\cdots\wedge \widetilde{\omega_{i_r}} \wedge\cdots\wedge \omega_{i_1} \\[0.3in]
&=& \ds 
	\sum_r (-1)^{p^2+r+1}\, a_{i_1\cdots i_r\cdots i_p, i_rk}\,
	\omega_k \wedge \omega_{i_p}\wedge\cdots\wedge \widetilde{\omega_{i_r}} \wedge\cdots\wedge \omega_{i_1}\,. \\
\end{array}$\\[0.1in]

On the other hand, 
$$ \delta \, d\, \omega \,=\, \delta\,\Big( a_{i_1\cdots i_p,j}\, \omega_j \wedge \omega_{i_p} \wedge\cdots\wedge \omega_{i_1}\Big) $$
We may let \,$\ds \beta \,=\, d\omega \,=\, \sum b_{i_1\cdots i_p i_{p+1}}\, \omega_{i_{p+1}}\wedge 
\omega_{i_p} \wedge\cdots\wedge \omega_{i_1}$\,. So we have\\[0.1in]
$\begin{array}{rcl}
\delta \, d\, \omega &=& \ds 
	\delta\,\Big( a_{i_1\cdots i_o,j}\, \omega_j \wedge \omega_{i_p} \wedge\cdots\wedge \omega_{i_1}\Big) \\[0.2in]
&=& \ds 
	\sum_{r=1}^p (-1)^{(p+1)^2+r+1}\, b_{i_1\cdots i_r\cdots i_p i_{p+1}, i_r}\, 
	\omega_{i_{p+1}} \wedge \omega_{i_p} \wedge\cdots\wedge \widetilde{\omega_{i_r}} \wedge\cdots\wedge \omega_{i_1} \\[0.1in]
&& \ds 
	+\, (-1)^{(p+1)^2+(p+1)+1}\, b_{i_1 \cdots i_p i_{p+1}, i_{p+1}}\,
	\omega_{i_p} \wedge\cdots\wedge \omega_{i_1} \\
\end{array}$ \\[0.1in]

Note that \\[0.1in]
$\begin{array}{rcl}
b_{i_1\cdots i_p i_{p+1}, j}\, \omega_j &=& \ds 
	db_{i_1\cdots i_p i_{p+1}} \,+\, \sum_{r,j_r} b_{i_1\cdots j_r\cdots i_p i_{p+1}} \, \omega_{j_r i_r}
	\,+\, b_{i_1\cdots i_p s}\, \omega_{s i_{p+1}} \\[0.1in]
&=& \ds 
	da_{i_1\cdots i_p, i_{p+1}} \,+\, \sum_{r,j_r} a_{i_1\cdots j_r\cdots i_p, i_{p+1}}\, \omega_{j_r i_r} 
	\,+\, a_{i_1\cdots i_p,s}\, \omega_{s i_{p+1}} \\[0.1in]
&=& \ds 
	a_{i_1\cdots i_p, i_{p+1}k} \, \omega_k \,.
\end{array}$ \\[0.1in]

So we have \, $b_{i_1\cdots i_p i_{p+1},j} \,=\, a_{i_1\cdots i_p, i_{p+1}j}$\, 
for every choice of \,$i_1, \cdots, i_p,\, i_{p+1} ,\, j$\,. \\[0.1in]
$\begin{array}{rcl}
\delta \, d\, \omega &=& \ds
	\sum_{r=1}^p (-1)^{p^2+r}\, a_{i_1\cdots i_r\cdots i_p, ji_r}\, 
	\omega_{j} \wedge \omega_{i_p} \wedge\cdots\wedge \widetilde{\omega_{i_r}} \wedge\cdots\wedge \omega_{i_1} \\[0.2in]
&& \ds 
	-\, \sum_j a_{i_1 \cdots i_p, jj}\,	\omega_{i_p} \wedge\cdots\wedge \omega_{i_1} \,.
\end{array}$ \\
\newpage 

As a result, the Laplacian of $\omega$ is found by \\[0.1in]
$\begin{array}{rcl}
\Delta\, \omega &=& \, -\, d\,\delta\, \omega \,-\, \delta\, d\, \omega \\[0.2in]
&=& \ds
	\sum_r (-1)^{p^2+r}\, a_{i_1\cdots i_r\cdots i_p, i_rk}\,
	\omega_k \wedge \omega_{i_p}\wedge\cdots\wedge \widetilde{\omega_{i_r}} \wedge\cdots\wedge \omega_{i_1} \\[0.1in]
&&\ds 
	+\, \sum_r (-1)^{p^2+r+1}\, a_{i_1\cdots i_r\cdots i_p, ji_r}\, 
	\omega_{j} \wedge \omega_{i_p} \wedge\cdots\wedge \widetilde{\omega_{i_r}} \wedge\cdots\wedge \omega_{i_1} \\[0.1in]
&& \ds 
	+\, \sum a_{i_1 \cdots i_p, jj}\,	\omega_{i_p} \wedge\cdots\wedge \omega_{i_1}  \\[0.2in]
&=& \ds 
	\sum a_{i_1 \cdots i_p, jj}\,	\omega_{i_p} \wedge\cdots\wedge \omega_{i_1} \\[0.1in]
&&\ds 
	+\, \sum_r (-1)^{p^2+r}\, \big(a_{i_1\cdots i_r \cdots i_p, i_rj} \,-\, a_{i_1\cdots i_r\cdots i_p, ji_r}\big)\,
	\omega_{j} \wedge \omega_{i_p} \wedge\cdots\wedge \widetilde{\omega_{i_r}} \wedge\cdots\wedge \omega_{i_1} \\[0.2in]
&=& \ds 
	\sum a_{i_1 \cdots i_p, jj}\,	\omega_{i_p} \wedge\cdots\wedge \omega_{i_1} \\[0.1in]
&&\ds +\, \sum_{r,s} a_{i_1\cdots i_r \cdots j_s \cdots i_p}\, R_{i_r j_r i_s j_s} \,
	\omega_{i_p} \wedge\cdots\wedge  \omega_{j_r} \wedge\cdots\wedge \omega_{i_1} \,.\\
\end{array}$  \\[0.1in]

For the $p$-form $\omega$, we let
$$ \nabla^\ast \nabla \,\omega\,=\, \sum a_{i_1 \cdots i_p, jj}\,	\omega_{i_p} \wedge\cdots\wedge \omega_{i_1} \,, $$
\mbox{}
$$ E(\omega) \,=\, \sum_{r,s} a_{i_1\cdots j_s \cdots i_p}\, R_{i_r j_r j_s i_s} \,
	\omega_{i_p} \wedge\cdots\wedge  \omega_{j_r} \wedge\cdots\wedge \omega_{i_1}\,. $$
Therfore, we have
\begin{equation}
\Delta \,\omega \,=\, \nabla^\ast \nabla \,\omega \,-\, E(\omega)\,. 
\end{equation}
\vspace*{0.4in}
% % % % % % % % % % % % % % % % % % % % % % % % % % % % % % % % % % %

\begin{center} {\sc 8. The Bochner formula}\end{center}
The Bochner formula states that if $\omega$ is a $p$-form,
$$ \omega \,=\, \sum a_{i_1 \cdots i_p}\,	\omega_{i_p} \wedge\cdots\wedge \omega_{i_1}\,,$$
then we have
\begin{equation}
	\nabla |\omega|^2 \,=\, 2<\Delta \omega\,,\,\omega> \,+\, 2\,|\nabla\,\omega|^2 \,+\, 
	2<E(\omega)\,,\,\omega>\,.
\end{equation}
\vspace*{4pt}

Before proving Equation (10), it is worth mentioning some basic results 
about applying the chain rule to covariant derivatives. Let $f$ be a smooth real-valued function
on the manifold $M$, and let $\phi: \mR \to \mR$. Note that we have
$$ \nabla\,\big(\phi \circ f\big) \,=\, \phi'(f)\,\nabla f$$
and so \,$ (\phi\circ f)_j \,=\, \phi'(f)\, f_j$\,. Moreover, \\

\newpage
$ \begin{array}{rcl}
(\phi \circ f)_{ij} &=& \ds 
	<\nabla_{e_i}\Big(\phi'(f)\,\nabla f\Big)\,,\, e_j> \\[0.1in]
&=& \ds
	e_i\big(\phi'(f)\big) \, <\nabla f\,,\, e_j> \,+\, \phi'(f)\,f_{ij} \\[0.1in]
&=& \ds 
	\phi''(f)\,f_i\,f_j \,+\, \phi'(f) \, f_{ij}\,. \\
\end{array} $ \\[0.1in]

If \,$\alpha \,=\, \sum a_j \, \omega_j$\, is a 1-form instead, and every \,$\phi_j:\mR \to \mR$\,
is a real-valued function, then we let 
$$ \beta \,=\, \sum \phi_j(a_j)\, \omega_j \,=\, \sum b_j\,\omega_j\,.$$

The covariant derivative of $\beta$ is obtained by\\[0.1in]
$ \begin{array}{rcl}
b_{j,k}\, \omega_k &=& \ds db_j \,+\, b_r\, \omega_{rj} \\[0.1in]
&=& \ds 
	d\big(\phi_j(a_j)\big) \,+\, \phi_r(a_r)\,\omega_{rj} \\[0.1in]
&=& \ds 
	\phi_j'(a_j)\, da_j \,+\, \phi_r(a_r)\, \omega_{rj} \\[0.1in]
&=& \ds
	\phi_j'(a_j)\, \Big(a_{j,l}\,\omega_l \,-\, a_s\,\omega_{sj}\Big) \,+\, \phi_r(a_r)\, \omega_{rj} \\[0.1in]
&=& \ds 
	a_{j,l}\,\phi_j'(a_j)\,\omega_l + \Big(\phi_r(a_r) \,-\, a_r\,\phi_j'(a_j)\Big)\, \omega_{rj} \\
\end{array} $ \\[0.1in]

Therefore, we have
$$ b_{j,k} \,=\, \phi_j'(a_j)\, a_{j,k} \,+\, \Big(\phi_r(a_r) \,-\, a_r\,\phi_j'(a_j)\Big)\, \Gamma_{kr}^j\,. $$
\vspace*{2pt}

Back to the Bochner formula, on the left hand side of Equation (10), 
$$ \Delta|\omega|^2 \,=\, \Delta\big(\sum a_{i_1\cdots i_p}^2 \big) \,=\, \sum (a_{i_1\cdots i_p}^2)_{jj}\,. $$
By our discussion above, 
$$ (a_{i_1\cdots i_p}^2)_{jj} \,=\, 2\big((a_{i_1\cdots i_p})_j\big)^2 \,+\, 2\,a_{i_1\cdots i_p}\,(a_{i_1\cdots i_p})_{jj}\,. $$
Therefore, 
$$ \Delta|\omega|^2 \,=\, 2\,\sum \big((a_{i_1\cdots i_p})_j\big)^2 \,+\, 2\, \sum a_{i_1\cdots i_p}\,(a_{i_1\cdots i_p})_{jj}\,. $$
\vspace*{2pt}

Here We add a remark that the terms \,$(a_{i_1\cdots i_p})_j$\, and \,$(a_{i_1\cdots i_p})_{jj}$\, 
are the components of the first and second covariant derivatives of the function $a_{i_1\cdots i_p}$ respectively.
In other words, 
$$ (a_{i_1\cdots i_p})_j \neq a_{i_1\cdots i_p,j} \man (a_{i_1\cdots i_p})_{jj} \neq a_{i_1\cdots i_p,jj}$$
in general. To make the equalities happen, in the following we fix a point $x$ on $M$.
Then, we choose an orthonormal frame \,$\big\{e_1,\cdots,e_n\big\}$\, around $x$ such that \\[0.1in]
(1) \quad $\nabla_{e_i}e_j \,=\, 0$\, at $x$ for all $i$, $j$; \\[0.1in]
(2) \quad $\nabla_{e_i} \nabla_{e_i}e_j \,=\, 0$\, at $x$ for all $i$, $j$. \\
\newpage 

In order to construct this orthonormal frame, we pick an orthonormal basis for $T_xM$ named by 
$\big\{E_1, E_2\cdots,E_n\big\}$\,. Consider the geodesic normal coordinates at $x$ 
such that $E_j$ is represented by \,$e_j\,=\,\big(0,\cdots, 1^{(j)},\cdots,0\big)$\, on $T_xM$. 
Any vector $E_j$ is parallel-transported from $x$ to another point $y=\exp_x(\bv)$ 
in the neighborhood through geodesics $\gamma(t) = \exp_x(t\,\bv) $ connecting $x$ and $y$. \\

In particular, for any pair of $E_i$ and $E_j$, 
$$ (\nabla_{E_i} E_j)(y) \,=\, {\bf 0} $$
whenever $y$ lies on the geodesic \,$\gamma_i(t) \,=\, \exp_x(t\,e_i)$\,. For any $E_k$, we have
$$ <\nabla_{E_i} E_j\,,\,E_k>\,=\, 0$$
at any point $y$ on $\gamma_i$. Therefore, 
$$ \begin{array}{rcl}
E_i\big(<\nabla_{E_i} E_j\,,\,E_k>\big) &=& 0 \\[0.1in]
<\nabla_{E_i}\nabla_{E_i} E_j\,,\,E_k> \,+\, <\nabla_{E_i} E_j\,,\,\nabla_{E_i}E_k> &=& 0 \\[0.1in]
<\nabla_{E_i}\nabla_{E_i} E_j\,,\,E_k> &=& 0
\end{array}$$
at $y$. Hence, \,$\nabla_{E_i}\nabla_{E_i} E_j\,=\,{\bf 0}$\, at $x$. \\[0.1in]

As an implication of properties (1) and (2), at the point $x$, \\[0.1in]
$\begin{array}{rcl}
a_{i_1\cdots i_p,j} &=& \ds 
	da_{i_1\cdots i_p}(e_j) \,+\, a_{i_1\cdots j_r\cdots i_p}\, \Gamma_{jj_r}^{i_r} \\[0.1in]
&=& \ds 
	da_{i_1\cdots i_p}(e_j)\,, \\[0.2in]
a_{i_1\cdots i_p,jj} &=& \ds 
	da_{i_1\cdots i_p,j}(e_j) \,+\, a_{i_1\cdots i_p,s}\,\omega_{sj}(e_j) 
	\,+\, \sum a_{i_1\cdots j_r \cdots i_p,j}\, \omega_{j_ri_r}(e_j) \\[0.1in]
&=& \ds 
	da_{i_1\cdots i_p,j}(e_j)\,.
\end{array} $ \\[0.1in]

So we have \,$(a_{i_1\cdots i_p})_j \,=\, da_{i_1\cdots i_p}(e_j) \,=\, a_{i_1\cdots i_p,j}$\, at $x$. \\[0.1in]
$\begin{array}{rcl}
(a_{i_1\cdots i_p})_{jj} &=& \ds 
	\big(d(a_{i_1\cdots i_p})_j\big)(e_j) \,+\, a_{i_1\cdots i_p,k}\, \omega_{kj}(e_j) \\[0.1in]
&=& \ds 
	d\Big(a_{i_1\cdots i_p,j} \,-\, a_{i_1\cdots j_r\cdots i_p}\, \Gamma_{jj_r}^{i_r}\Big)(e_j) \\[0.1in]
&=& \ds 
	d(a_{i_1\cdots i_p,j})(e_j) \,-\, da_{i_1\cdots j_r\cdots i_p}(e_j)\, \Gamma_{jj_r}^{i_r} 
	\,-\, a_{i_1\cdots j_r\cdots i_p}\, d\Gamma_{jj_r}^{i_r}(e_j) \\
\end{array} $ \\[0.1in]

Note that \\[0.1in]
$\begin{array}{rcl}
d\Gamma_{jj_r}^{i_r}(e_j) &=& \ds e_j\big(<\nabla_{e_j} e_{j_r}\,,\, e_{i_r}>\big) \\[0.1in]
&=& \ds 
	<\nabla_{e_j}\nabla_{e_j} e_{j_r}\,,\, e_{i_r}> \,+\, <\nabla_{e_j} e_{j_r}\,,\, \nabla_{e_j}e_{i_r}> \\[0.1in]	
&=& \ds 
	0 \quad \mbox{at }x\,. 
\end{array}$ \\
\newpage

Therefore, \,$(a_{i_1\cdots i_p})_{jj} \,=\, a_{i_1\cdots i_p,jj}$\, at $x$. 
Finally, we may justify the Bochner formula as follows. \\[0.1in]
$\begin{array}{rcl}
\Delta |\omega|^2 &=&\ds 
	2\,\sum a_{i_1\cdots i_p,j}^2 \,+\, 2\,\sum a_{i_1\cdots i_p}\, a_{i_1\cdots i_p,jj} \\[0.2in]
&=& \ds 
	2\, |\nabla \omega|^2 \,+\, 2\,<\nabla^\ast\nabla \omega \,,\, \omega> \\[0.1in]
&=& \ds 
	2\, |\nabla \omega|^2 \,+\, 2\,<\Delta \omega \,,\, \omega> \,+\, 2\,<E(\omega) \,,\, \omega>
\end{array}$ 

\vfill
\centering{\sc -end-}
% % % % % % % % % % % % % % % % % % % % % % % % % % % % % % % % % % %
\end{document}